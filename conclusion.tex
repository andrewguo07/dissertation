\chapter{Conclusion}
We conclude with some discussion of open directions.

$VLR scrambling$

Finding an optimal protocol that saturates the Lieb-Robinson bounds for $\al \le d$ is still an open question, even though we've proven that the bounds are tight for many-site signaling.
Also, recent work has shown that the scrambling time is better bounded by the so-called Frobenius bound, which has been shown to yield tighter bounds than the Lieb-Robinson bound for the Frobenius norm of the unequal-time comutator (which can be related to the out-of-time-order correlator). Indeed, a hierarchy of linear light cones has been shown in which the Frobenius bound yields a finite scrambling speed at a smaller value of $\alpha$ than the Lieb-Robinson bound does for signaling, thus clearly delineates these types of bounds \cite{Tran2020hierarchylinearlightcones}.
For higher dimensions, the conjectured Frobenius light cone is $t$\,$=$\,$\W{r^{\alpha-d}}$ for $d$\,$<$\,$\al$\,$<$\,$d+1$ \cite{Chen2021Frobenius}.
If this generalization of the Frobenius bound were to hold, our lower bounds on the circuit complexity of QFT and fanout in \cref{ch:qfo} would immediately generalize.

%QFO

We have derived our lower bounds on $t_\textrm{QFT}$ under the assumption that the first and last qubits of the QFT are separated by a distance of $r$\,$=$\,$\Theta(n^{1/d})$.
However, other mappings of computational qubits to lattice qubits could potentially lead to faster implementations.
For example, consider the mapping onto a one-dimensional chain of qubits wherein the second half of the chain is interleaved in reverse order with the first half \footnote{Specifically, for even $n$, this mapping is defined by $q_i\mapsto q_{2i-1}$ for $i \le n/2$ and $q_i\mapsto q_{2(n-i+1)}$ for $i > n/2$.}.
Applying the QFT to a product state in this layout results in a state with two-qubit correlations that decay exponentially in the distance between the qubits.
In this case, our lower bound techniques cannot rule out the possibility of $t_\textrm{QFT}$\,$=$\,$o(n)$ for short-range interacting Hamiltonians.
This suggests that $t_\textrm{QFT}$ could depend strongly on qubit placement.
Given that the QFT is typically used as a subroutine for more complex algorithms, it may not always be possible to reassign qubits without incurring costs elsewhere in the circuit.
Still, it would be interesting to investigate whether careful qubit placement could yield a faster QFT.

% OCC
In this work, we have  generalized the proof of Lieb-Robinson bounds from closed system dynamics to Markovian evolution (\textit{a priori}, such bounds need not exist for Markovian dynamics).
However, our bounds only depend on interaction range and the dimension of the lattice.
Any bound that only depends on these two inputs cannot be tighter than the corresponding closed-system Lieb-Robinson bound, since the latter is a special case of former.
As such, the saturating protocols for closed systems \cite{Tran2020hierarchylinearlightcones,Tran2021a} can be used to saturate open Lieb-Robinson bounds such as those uncovered in  \cref{lemma:LRbound_truncated}.
In the future, it would be interesting to add another degree of freedom into formulations of open Lieb-Robinson bounds: the dissipative gap. (Some progress has been made in showing that Lieb-Robinson velocities can get tighter in dissipative systems \cite{descamps2013}.)
In principle, it might be possible to derive Lieb-Robinson bounds that reduce to closed-system ones when the dissipative gap is zero, and get tighter in the presence of non-zero dissipation.
Then one can develop protocols that saturate the dissipative-gap-dependent bounds.
Another question in this direction is whether the conditional evolution generated via a non-Hermitian Hamiltonian can also exhibit a dissipative-gap-dependent Lieb-Robinson bound that reduces to the conventional one in the dissipationless limit.

Lieb-Robinson bounds can be used  to prove area-law entanglement scaling in the ground state of one-dimensional systems with local interactions \cite{Hastings2007}. This result helps to rigorously justify the validity of the matrix-product state  ansatz for the ground state of such systems.  For closed systems with power-law interactions, Lieb-Robinson bounds can be used to further extend area-law scaling to certain broad classes of systems \cite{Gong2017}. Do the results presented in this paper have similar implications for area-law scaling of the steady state? This would have direct implications for the matrix-product operator ansatz in modeling open systems.

Finally, the Lieb-Robinson-type bounds we proved apply for the operator, or $\infty$-norm.
However, there exists a hierarchy of Lieb-Robinson-like bounds that have the potential to be tighter for certain information processing tasks such as scrambling and transferring a quantum state of a local subsystem without knowledge of the initial state of the rest of the system.
These bounds can use other norms such as the Frobenius norm defined by $\|O\|_F = \sqrt{\Tr{O^\dag O}}$ \cite{Tran2020hierarchylinearlightcones,Kuwahara2020aPolynomialGrowthOutoftimeorder,Yin2020ScramblingAlltoall,Chen2021Frobenius} or apply to free-particle systems \cite{Guo2019,Tran2020hierarchylinearlightcones}.
It would be interesting to generalize these bounds to open systems as well.
