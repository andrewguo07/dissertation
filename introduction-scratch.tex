\chapter{Introduction}

Quantum information undergoes many transformations. It is ;possible to create non-local quantum states
Power-law systems are exciting due to their ability to act as quantum sensors, clocks. Experimental control in the progress of AMO systems have allowed for the creation of multiescale entangled states in such systems. The create nof such entangled states quickly is enabled by power-law interactions.

%%% These interactions can provide a speed-up over short-range systems with architectural constraints. %%%

The standard circuit model assumes one may apply single-qubit and two-qubit gates from a standard gate set on arbitrary non-overlapping subsets of the qubits.
However, the assumption of being able to directly apply interactions between two arbitrarily distant qubits does not hold in practice for large quantum computing architectures \cite{Monroe2014,Linke2017,Bapat2018,Childs2019c,Schoute2022}.
Mapping these circuits to restricted architectures inevitably leads to overheads and potentially even different asymptotic scaling.

Systems with power-law interactions, however, present an opportunity for realizing speed-ups.
Specifically, for a lattice of qubits in $D$ dimensions, the interaction strengths between pairs of qubits separated by a distance $r$ are weighted by a power-law decaying function $1/r^\alpha$. These long-range interactions are native to many experimental quantum systems and have attracted interest as potential resources for faster quantum information processing. Examples of long-range interactions include dipole-dipole and van der Waals interactions between Rydberg atoms~\cite{Saffman2010,Weimer2012}, and dipole-dipole interactions between polar molecules~\cite{Yan2013} and between defect centers in diamond~\cite{Yao2012,Weimer2012}.

Recently, Refs.~\cite{Eldredge2017,Guo2020,Tran2020,kuwaharaStrictlyLinearLight2020} gave protocols that take advantage of power-law interactions to quickly transfer a quantum state across a lattice.
As we will show in Section [], it is also possible to leverage the power of these interactions to implement quantum gates asymptotically faster than is possible with finite-range interactions.

%%% Open systems clustering of correlations introduction

Correlations in many-body quantum systems have many uses, from indicating the presence of criticality to measuring the spread of entanglement. In particular, the fact that correlations decay rapidly in some states, a phenomenon referred to as the “clustering of correlations,” has been used to prove the pivotal \emph{area law} for the entanglement entropy in 1D systems. This result has had significant implications for the development of classical tensor network algorithms for simulating these quantum systems \cite{Hastings07}.

We also restore a notion of locality by proving two Lieb-Robinson bounds, which constrain information propagation in many-body quantum systems, for systems whose dynamics can be modeled by a Liouvillian master equation. We then use these bounds to prove that correlations decay rapidly in the steady-states of these open long-range systems, thus marking the first step towards proving the entanglement area law in these states.
