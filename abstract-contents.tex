Quantum many-body systems with long-range interactions--—e.g. those that decay as a power-law in the distance between particles—--have received significant interest of late due to their promise as quantum information processors. Most of these systems operate in a non-relativistic regime, however, where the lack of an absolute speed limit stymies attempts to bound signal propagation. To restore a notion of locality, Elliot Lieb and Derek Robinson proved bounds that constrained the rates of information transfer. These bounds led to light-cone-like regions outside which signals would be exponentially suppressed. In long-range systems, however, the question of whether or not there exist bounds which--—for sufficiently rapidly decaying interaction strength—--yield exactly linear light cones has remained open until recently. 

In this thesis, I will describe recent results pertaining to the rates of information propagation in power-law-interacting systems. First, I will present results for the regime of ``strongly long-range'' interactions, for which velocities can grow unboundedly with system size. I will present Lieb-Robinson-type bounds for these systems and also outline a protocol that can transfer quantum states as fast as these bounds will allow. I will discuss the implications of these bounds for scrambling.

The second part of the thesis will study how protocols for transferring quantum states quickly can be used to perform multiple-qubit gates. In particular, I will demonstrate how the power of long-range interactions allows one to implement the unbounded fanout gate asymptotically faster than systems with local interactions. This result also implies the hardness of simulating the dynamics of long-range systems evolving for superlogarithmic times, and demonstrates the potential for insights from quantum many-body physics to lead to a more powerful toolbox for quantum computation.

Finally, I will address the question of fundamental speed limits in quantum systems that are open to the environment. A priori, it may seem surprising that such speed limits may exist, since non-unitary processes may break locality constraints. However, we show that under certain assumptions such as linearity and Markovianity of the bath, one can restore a notion of locality using Lieb-Robinson-type bounds. We use the resulting bounds to constrain the entanglement structure of the steady states of open long-range systems, a first step towards proving the area law for such systems. 
