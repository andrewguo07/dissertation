Quantum many-body systems with long-range interactions--—e.g. those that decay as a power-law in the distance between particles—--have received significant interest of late due to their promise as quantum information processors. In non-relativistic models of spin systems, the lack of an absolute speed limit could in principle allow for instantaneous signal propagation. To restore a notion of locality, Elliot Lieb and Derek Robinson proved bounds that constrained the rates of information transfer. These bounds led to light-cone-like regions outside which signals would be exponentially suppressed. In long-range systems, however, the question of whether or not there exist bounds which--—for sufficiently rapidly decaying interaction strength—--yield exactly linear light cones has remained open until recently. 

In this thesis, I will describe recent results mapping out the phase diagram of light cones for power-law-interacting systems, focusing on the regime of ``strongly long-range'' interactions, which include the all-to-all limit. I will also outline a protocol that can transfer quantum states as fast as these bounds will allow. This state-transfer protocol has numerous practical applications, from creating large entangled states for metrology to implementing multi-qubit quantum gates asymptotically faster than systems with local interactions. This final result demonstrates the potential for insights from quantum many-body physics to lead to a more powerful toolbox for quantum computation.

Finally, I will address the question of such bounds in open quantum systems. A priori, it may seem surprising that a speed limit may exist, since non-unitary processes may break locality constraints. However, we show that assuming a Markovian bath, a notion of locality can be restored. We use the resulting bounds to prove that correlations decay rapidly in the steady states of open long-range systems.
