Introduction:

A finite signaling speed has been shown for $\al > 2d+1$. Indeed the question of the fastest signaling speed has largely been settled for $\al > d$ \cite{Tran...}. There, a set of bounds has been shown to be saturable up to some sub polynomial factors. However, for the regime of $\al \le d$, the question of the fastest signaling time still remains open.

Recent work has shown that the scrambling time is better bounded by the so-called Frobenius bound, which has been shown to yield tighter bounds than the Lieb-Robinson bound in 1D systems. Indeed, a hierarchy of linear light cones has been shown in which the Frobenius bound yields a finite scrambling speed at a smaller value of $\alpha$ than the Lieb-Robinson bound does for signaling, thus clearly delineates these types of bounds.

Addendum:

A recent work by Yin and Lucas have shown a tighter scrambling time lower bound of $t_{\text{sc}}\gtrsim 1/N^{1/2}$ for all-to-all interactions, thus improving over our result $t_{\text{sc}}\gtrsim \log(N)/N$. Their bound is tight in a sense for scrambling since they are able to show a minimal model that gives a fast scrambler.

The free-particle state-transfer protocol has been used to demonstrate a separation between gate-based quantum routing and Hamiltonian-based routing (with fast access to ancillae) \cite{Bapat2022}. As such, it is useful for demonstrating the power of the Hamiltonian model as well as the strength of non-local interactions.
