% \usepackage[latin1]{inputenc}
\usepackage{graphicx}
\usepackage{amsmath}
\usepackage{amsthm}
\usepackage{amsfonts,dsfont}
\usepackage{bm}
\usepackage{layout}
\usepackage{float}
\usepackage{amssymb}%
% \usepackage[margin=0.75in]{geometry}
\usepackage{color}
\usepackage{soul}
\usepackage{mathtools}
\usepackage{varwidth}
% \usepackage[colorlinks=true,citecolor=blue]{hyperref}
\usepackage{subfigure}
\usepackage{xr} %allow cross-referencing across files
\usepackage{times}
\usepackage{algpseudocode}
\usepackage{algorithm}
\usepackage{mathtools}
\usepackage{qcircuit}
\usepackage{comment}
\usepackage{complexity}

\theoremstyle{definition}
% \newtheorem{theorem}{Theorem}
\newtheorem{claim}{Claim}
\newtheorem{problem}{Problem}
% \newtheorem{lemma}{Lemma}
\newtheorem{definition}{Definition}
\newtheorem{conjecture}{Conjecture}
% \newtheorem{corollary}{Corollary}
\usepackage[capitalise,compress]{cleveref}
\newcommand{\avg}[1]{\left \langle #1 \right\rangle}
\newcommand{\corrc}[1]{\left \langle #1 \right\rangle_c}
\newcommand{\ket}[1]{\left | #1 \right\rangle}
\newcommand{\oket}[1]{\left | #1 \right)}
\newcommand{\bra}[1]{\left \langle #1 \right |}
\newcommand{\half}{\frac{1}{2}}
\newcommand{\smallfrac}[2][1]{\mbox{$\textstyle \frac{#1}{#2}$}}
\newcommand{\Tr}{\mathrm{Tr}}
\renewcommand{\Re}{\mathrm{Re}}
\newcommand{\braket}[2]{\left\langle #1|#2\right\rangle}
\newcommand{\bracket}[3]{\left\langle #1|#2|#3\right\rangle}
\newcommand{\proj}[1]{\ket{#1}\bra{#1}}
\newcommand{\abs}[1]{\left | #1 \right|}
\renewcommand{\epsilon}{\varepsilon}
\renewcommand{\O}[1]{ O\left(#1\right)}
%\DeclarePairedDelimiter{\norm}{\lVert}{\rVert}
\newcommand{\norm}[1]{\left\|#1\right\|}
\newcommand{\lnorm}[1]{\left\|#1\right\|_l}
\newcommand{\comm}[1]{\left[#1\right]}
% \newcommand{\C}{\mathcal{C}}
\newcommand{\Laplace}{\mathcal{L}}
% \newcommand{\red}[1]{{\color{red} #1}}
\newcommand{\arrowto}{$\Rightarrow$ }
\newcounter{para}
\newcommand{\dist}{\textrm{dist}}
\newcommand\observation{ \par\emph{Observation \refstepcounter{para}\thepara.\space}}
\makeatletter
\newcommand*\bigcdot{\mathpalette\bigcdot@{.5}}
\newcommand*\bigcdot@[2]{\mathbin{\vcenter{\hbox{\scalebox{#2}{$\m@th#1\bullet$}}}}}
\newcommand{\U}[2]{U^{#1}_{#2}}
\newcommand{\Udag}[2]{\left(U^{#1}_{#2}\right)^\dag}
\newcommand{\B}{\mathcal B}
\newcommand{\const}{\text{const.}}
%\renewcommand{\labelitemi}{--}
\makeatother
\newcommand{\nmax}{\omega_*}
\newcommand{\qmax}{q_{\text{max}}}
\newcommand{\ad}{\text{ad}}
\newcommand{\Hpl}{\mathbb{H}_{\alpha}}

\newcommand{\tr}{\text{Tr}}
% \newcommand{\mL}{\mathcal{L}}
% \newcommand{\mT}{\mathcal{T}}
% \newcommand{\mQ}{\mathcal{Q}}

%Packages/Commands Added by Adam%
\usepackage{braket}
\newcommand{\Z}{\mathbb{Z}}
\newcommand\numberthis{\addtocounter{equation}{1}\tag{\theequation}}
\usepackage{array}
\newcolumntype{L}{>{$}l<{$}} % math-mode version of "l" column type
\newcolumntype{C}{>{$}c<{$}} % math-mode version of "c" column type
\newcolumntype{R}{>{$}r<{$}} % math-mode version of "r" column type

\usepackage{xcolor}
\usepackage{mathtools}


% Commands added by Andrew
\usepackage{physics}
\usepackage{tikz}

\newcommand{\eps}{\ensuremath{\varepsilon}}
\newcommand{\vphi}{\ensuremath{\varphi}}
\newcommand{\ra}{\ensuremath{\rightarrow}}
\newcommand{\la}{\ensuremath{\leftarrow}}
\newcommand{\ds}{\ensuremath{\displaystyle}}
\newcommand{\vn}{\ensuremath{\varnothing}}
\newcommand{\al}{\ensuremath{\alpha}}
\newcommand{\lam}{\ensuremath{\lambda}}
\newcommand{\Lam}{\ensuremath{\Lambda}}
\renewcommand{\norm}[1]{\left\|#1\right\|}


\renewcommand{\mL}{\L}
\newcommand{\mT}{\mathcal{T}}
\newcommand{\mQ}{\mathcal{Q}}

\newcommand{\Ra}{\ensuremath{\: \Rightarrow \:}}

% Make nice displaystyle big-O,Ω,θ notations without the weird spaces
\DeclarePairedDelimiter\parentheses{\lparen}{\rparen}
\renewcommand{\O}[1]{\mathcal{O}\parentheses*{#1}}
\renewcommand{\W}[1]{\Omega\parentheses*{#1}}
\newcommand{\Th}[1]{\Theta\parentheses*{#1}}

\newcommand{\J}{\mathcal{J}}
\renewcommand{\L}{\mathcal{L}}
\renewcommand{\C}{\mathcal{C}}
\newcommand{\Wlog}{Without loss of generality}
\newcommand{\opket}[1]{|#1)}

% GuoVLR shortcuts
\newcommand{\vac}{0}
\newcommand{\td}{\tilde}

% QFO shortcuts
\newcommand{\tghz}{t_{\text{GHZ}}}
\newtheorem*{theorem*}{Theorem}
\newtheorem*{lemma*}{Lemma}
% declaration of a new "parallel for" block
\algblock{ParFor}{EndParFor}
% customising the new block
\algnewcommand\algorithmicparfor{\textbf{parfor}}
\algnewcommand\algorithmicpardo{\textbf{do}}
\algnewcommand\algorithmicendparfor{\textbf{end\ parfor}}
\algrenewtext{ParFor}[1]{\algorithmicparfor\ #1\ \algorithmicpardo}
\algrenewtext{EndParFor}{\algorithmicendparfor}

% Ceiling and floor operators
\DeclarePairedDelimiter\ceil{\lceil}{\rceil}
\DeclarePairedDelimiter\floor{\lfloor}{\rfloor}

% Shortcuts for section labels and organization
\newcommand{\dash}{\textemdash}
\newcommand{\etal}{\emph{et al.}}
\newcommand{\sect}[1]{\emph{{#1}}.\dash}
\newtheorem{theorem}{Theorem}
\newtheorem{lemma}{Lemma}
\newtheorem{corollary}{Corollary}
\newtheorem{proposition}{Proposition}

% Clever ref
\usepackage[capitalise,compress]{cleveref}
\crefname{section}{Sec.}{Secs.}
\crefname{section}{Section}{Sections}
\crefrangelabelformat{equation}{\textup{(#3#1#4)}--\textup{(#5#2#6)}}

% Editorial commands
\newcommand{\red}[1]{\textcolor{red}{[#1]}}
