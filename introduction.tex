%!TEX root = body.tex
\chapter{Introduction}

%Title
% Many-body entanglement dynamics and computation in quantum systems with power-law interactions
At the heart of every quantum computer lies a many-body quantum system. These systems can inhabit a rich and complex class of states with exotic and interesting properties in their own right. One of the fundamental properties that delineate them from classical systems is their ability to experience entanglement. Entanglement allows quantum systems to experience an extra level of correlations that extend beyond classical probability theory. These correlations provide a source of \emph{quantum} information and are what enable the performance of quantum computing. Indeed, a quantum computation can be viewed as the dynamics of a many-body system whose evolution to an entangled state encodes a computational problem. As such, the rate at which a many-body system can generate entanglement directly informs how quickly this computation can be performed in practice.

Most many-body systems relevant to modern quantum technologies can be viewed to operate in a non-relativistic regime,
%[citation needed?]
where typical velocities of information propagation are far below the threshold set by the speed of light. In this regime, an absolute speed limit is lacking due to the absence of causality inherent in the Schr\"odinger equation.
As such, a fundamental question in quantum many-body physics is, \emph{what are the fastest rates at which entanglement can spread in such systems?}

The first bounds on these rates were shown by Elliot Lieb and Derek Robinson in 1972 \cite{LR}.
Since then, much progress has been made on sharpening these bounds \cite{ChenLucas2021graphtheory,WangHazzard2020} and proving them for specific classes of systems \cite{Tran2019a,Chen2019,kuwaharaStrictlyLinearLight2020,Tran2021b}.
In addition to bounding the rate of entanglement generation, these bounds are also connected to a diverse array of phenomena, including the decay of correlations in the ground state \cite{Hastings2006}, generation of topological order \cite{Bravyi2006, Bravyi2010}, efficiency of classical/quantum simulation \cite{Osborne2006,Tran2019a}, hardness of bosonic sampling tasks \cite{Deshpande2018}, heating rates in periodically driven Floquet systems \cite{Abanin2015,Tran2019b}, and signatures of quantum chaos \cite{Lashkari2013,Guo2019}.

The ability of quantum computers to generate entanglement is central to their ability to achieve speed-ups in problems that are believed to be intractable for classical computers.
Solving hard problems quickly has been the selling point of quantum computers since 1994, when Peter Shor discovered his algorithm for fast integer factorization \cite{Shor1997}.
While in practice the computational speed of a quantum computer is inherently determined by parameters of the hardware that realizes the computer, the ``software'' can also play an important role.
In particular, the choices of algorithms and protocols used to perform the various gates and subroutines in the quantum circuit can affect the asymptotic runtimes.
% These protocols are highly dependent on the mapping between the circuit-level system and the system-level hardware.  the quantum circuits, which can affect their asymptotic runtimes.
% Choosing the right sets of gates to match the native hardware and takes best advantage of the limited resources is an important problem.
It can therefore be advantageous to study the theoretically optimal rates of entanglement generation in general models that are universal to all quantum computers, regardless of the underlying hardware.
% As such, it's important to study the theoretically optimal rates at which entanglement can be generated in these systems---to ensure that the computation takes the best advantage of the limited resources.

In terms of quantum hardware, today's quantum devices are indeed quite limited. They contain small numbers of qubits that decohere quickly. Furthermore, many of the prevalent models also rely on restricted qubit layouts such as a 2D planar grid architecture \cite{Arute2019}, whereas the standard circuit model of quantum computing assumes one may apply single-qubit and two-qubit gates on arbitrary non-overlapping subsets of the qubits.
Since the assumption of being able to directly apply interactions between two arbitrarily distant qubits does not hold in practice for large quantum computing architectures \cite{Monroe2014,Linke2017,Bapat2018,Childs2019c,Bapat2022}, computational overheads are incurred when mapping circuits to these restricted connectivities.
These overheads affect the asymptotic scaling of quantum algorithms and can even negate certain types of quantum advantage\footnote{for instance, Grover-type speed-ups}.
As such, these hardware constraints motivate the need to study both novel architectures as well as new ways of  generating entanglement quickly.

In terms of novel architectures for quantum computing, systems that possess longer-range interactions provide promising candidates.
In particular, power-law interactions---those that decay as a power-law $1/r^\alpha$ in the distance  $r$ between particles, for some $\alpha > 0$---provide a natural way of augmenting the power of quantum systems.
These not-so-local interactions are native to many experimental quantum systems and include dipole-dipole and van der Waals interactions between Rydberg atoms~\cite{Saffman2010,Weimer2012}, dipole-dipole interactions between polar molecules~\cite{Yan2013}, defect centers in diamond~\cite{Yao2012,Weimer2012}, and magnetic atoms~\cite{Fraxanet2022}.
Such systems have attracted interest due to their ability to act as quantum sensors \cite{Foss-Feig15} and clocks, in addition to their potential as resources for faster quantum information processing.

Recently, Refs.~\cite{kuwaharaStrictlyLinearLight2020,Eldredge2017,Guo2020,Tran2021a} gave protocols that take advantage of power-law interactions to quickly transfer a quantum state across a lattice.
As we will show in \cref{ch:qfo}, it is also possible to leverage the power of these interactions to implement quantum gates asymptotically faster than is possible with finite-range interactions.
Furthermore, the Lieb-Robinson bounds discussed earlier, which bound the rate of transferring quantum states or engineering many-body entangled states, can also lead to new tools for lower-bounding the runtimes of quantum algorithms.
% And using many-body Hamiltonians, it may be possible to engineer protocols to transfer quantum states quickly---which can be used for quantum routing---or to produce many-body entangled states quickly.
% As we show in this thesis, these entangled states can be used as resources for performing certain multiqubit quantum gates quickly.
These two applications combined demonstrate the power of many-body physics to enhance the computational toolkit for quantum information science.

% In \cref{ch:vlr}, we provide one such demonstration of this synthesis of perspectives. We provide a set of matching upper and lower bounds for the time it takes to transfer quantum states in systems that can be mapped to free bosons or fermions hopping on a $d$-dimensional lattice with $1/r^{\alpha}$ hopping strength. Specifically, for systems with $N$ lattice sites, we prove a Lieb-Robinson-type bound of on the time required to transfer a single free boson/fermion $t \gtrsim N^{\alpha/d}/\sqrt{N}$ for $\al < d/2$ and show that it can be saturated by a new quantum state transfer protocol. We also prove a bound for many-site signaling (from one site to an extensive part of the system) that can be saturated. This bound leads to a bound on scrambling of $t_\text{sc}\gtrsim N^{\alpha/d}/N$, which generalizes the result in Ref.~\cite{Lashkari13} of $t_\text{sc} \gtrsim 1/N$ to all $\alpha<d$.

In \cref{ch:vlr}, we provide one such demonstration of this synthesis of perspectives. We provide a set of matching upper and lower bounds for the time it takes to transfer quantum information in systems that can be mapped to free bosons or fermions hopping on a $d$-dimensional lattice with $1/r^{\alpha}$ hopping strength. Specifically, for strongly long-range systems with $\al < d/2$, we prove a Lieb-Robinson-type bound on the time required to transfer a single free boson/fermion and show that it can be saturated by a protocol to transfer an unknown quantum state. We also prove a bound for sending quantum information from one site to an extensive part of the system and show that it can likewise be saturated. This bound leads to a bound on the time it takes for the system to scramble quantum information, which generalizes the fast-scrambling result in Ref.~\cite{Lashkari2013} to all strongly long-range systems.

In \cref{ch:qfo}, we demonstrate the power of long-range systems to obtain quantum speed-ups for quantum computation. In particular, we propose a method to engineer power-law interacting Hamiltonians to quickly generate a multi-qubit quantum gate known as the unbounded fanout gate and that it is able to achieve asymptotic speed-ups over short-range systems for all $\al \le 2d+1$. As an application of our protocol, we show that simulating long-range systems with $\al \le 2d$ for polylogarithmic times or longer is classically intractable, if factoring is classically hard. As a complement to our upper bounds on the fanout time, we also develop a technique that allows us to prove the tightest-known lower bounds for the time required to implement the QFT and unbounded fanout in general lattice architectures.

Finally, in \cref{ch:occ}, we consider the dynamics of information transfer in long-range systems that are coupled to an environment. These ``open'' systems evolve non-unitarily and model realistic experimental systems, which can suffer noise and decoherence.
We prove open-system Lieb-Robinson-type bounds on the dynamics of systems coupled to Markovian environments and use these bounds to constrain the entanglement structure of the states of these systems in the limit of infinitely long evolution times---i.e. of their ``steady states.'' In particular, we prove bounds on the decay of spatial correlations in these steady states. This result may serve as a first step towards establishing an area-law scaling of entanglement for these systems, similar to what was done in Ref.~\cite{Gong2017} for the closed case. % better motivation here of why area-law is important!

\newpage
\section*{Citations to Previously Published Work}
Most of the work appearing in this thesis is either published or has appeared on the preprint server arXiv. We mention these here.

\begin{itemize}
    \item Chapter 2: ``Signaling and scrambling with strongly long-range interactions''
    A. Y. Guo, M. C. Tran, A. M. Childs, A. V. Gorshkov, and Z-X. Gong, Phys. Rev. A 102, 010401 (2020).

    Further work related to Lieb-Robinson bounds and state transfer in power-law systems appeared in the following works:
    \begin{itemize}
        \item ``Locality and digital quantum simulation of power-law interactions,'' M. C. Tran, A. Y. Guo, Y. Su, J. R. Garrison, Z. Eldredge, M. Foss-Feig, A. M. Childs, A. V. Gorshkov, Phys. Rev. X 9, 031006 (2019).
        \item ``Locality and Heating in Periodically Driven, Power-law Interacting Systems,'' M. C. Tran, A. Ehrenberg, A. Y. Guo, P. Titum, D. A. Abanin, A. V. Gorshkov, Phys. Rev. A 100, 052103 (2019)
        \item ``Hierarchy of Linear Light Cones with Long-Range Interactions'', M. C. Tran, C.-F. Chen, A. Ehrenberg, A. Y. Guo, A. Deshpande, Y. Hong, Z.-X. Gong, A. V. Gorshkov, and A. Lucas, Phys. Rev. X 10, 031009 (2020).
        \item ``Optimal State Transfer and Entanglement Generation in Power-law Interacting Systems,'' M. C. Tran, A. Y. Guo, A. Deshpande, A. Lucas, A. V. Gorshkov, Phys. Rev. X 11, 031016 (2021).
        \item ``The Lieb-Robinson light cone for power-law interactions.'' M. C. Tran, A. Y. Guo, C. L. Baldwin, A.Ehrenberg, A. V. Gorshkov, A. Lucas, Phys. Rev. Lett. 127.160401 (2021)
        \item ``Disordered Lieb-Robinson bounds in one dimension.'' C. L. Baldwin, A. Ehrenberg, A. Y. Guo, A. V. Gorshkov, arXiv:2208.05509
    \end{itemize}

    Further work related to scrambling in power-law systems appeared in the following works:
    \begin{itemize}
        \item ``The operator Lévy flight: light cones in chaotic long-range interacting systems.'' T. Zhou, S. Xu, X. Chen, A. Y. Guo, and B. Swingle, Phys. Rev. Lett. 124, 180601 (2020).
        \item ``Hydrodynamic theory of scrambling in chaotic long-range interacting systems,'' T. Zhou, A. Y. Guo, S. Xu, X. Chen, and B. Swingle, arXiv: 2208.01649
    \end{itemize}

    \item Chapter 3: ``Implementing a fast unbounded quantum fanout gate using power-law interactions.'' A. Y. Guo, A. Deshpande, S.-K. Chu, Z. Eldredge, P. Bienias, D. Devulapalli, Y. Su, A. M. Childs, and A. V. Gorshkov, arXiv: 2007.00662

    \item Chapter 4: ``Clustering of steady-state correlations for open systems with long-range interactions.'' A. Y. Guo*, S. Lieu*, M. C. Tran, A. V. Gorshkov, arXiv: 2110.15368 *co-first-authors
\end{itemize}
%
%
% See also
%
%
%
% vector is stack-allocated, but elements are heap-allocated (???)
